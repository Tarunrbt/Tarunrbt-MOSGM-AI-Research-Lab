\documentclass[11pt,twocolumn,a4paper]{article}
\usepackage[utf8]{inputenc}
\usepackage{amsmath, amssymb, amsfonts}
\usepackage{graphicx}
\usepackage{geometry}
\usepackage{xcolor}
\usepackage{hyperref}
\usepackage{cite}

% --- MOSGM Custom Styling ---
\geometry{margin=2cm}
\hypersetup{colorlinks=true, linkcolor=blue, citecolor=red, urlcolor=cyan}

\title{\textbf{MOSGM-I: A Gradient-Dependent Spacetime Response Model as a Mechanistic Basis for MONDian Scaling}}
\author{\textbf{Tarunrbt} \\ \textit{Lead Investigator, MOSGM-AI Research Lab}}
\date{February 2026}

\begin{document}

\maketitle

\begin{abstract}
We present the Matter-Organized Spacetime Gradient Model (MOSGM-I), a framework where gravitational coupling is an emergent response to baryonic density gradients. While phenomenologically consistent with Modified Newtonian Dynamics (MOND) in low-gradient regimes, MOSGM-I identifies a unique "Goldilocks Zone" at high-gradient galactic edges. This paper, developed through a human-led multi-AI orchestrated audit (DeepSeek, Grok, Gemini), establishes the formal degeneracy limits of the model and provides testable predictions for the Square Kilometre Array (SKA) era.
\end{abstract}

\section{Introduction}
The persistent success of MOND in fitting galactic rotation curves remains a fundamental challenge to the $\Lambda$CDM paradigm. However, the lack of a mechanistic origin for the acceleration scale $a_0$ suggests that MOND may be a limiting case of a deeper theory. MOSGM-I proposes that spacetime curvature is not merely a function of mass, but a non-linear response to its spatial organization ($\nabla \rho$).

\section{Mathematical Framework}
The core postulate of MOSGM-I is the modification of the gravitational potential response $\Phi$ as a function of the baryonic gradient:
\begin{equation}
    G_{eff} = G_N \left[ 1 + \mathcal{F}\left( \frac{|\nabla \rho|}{\rho_c} \right) \right]
\end{equation}
Where $\mathcal{F}$ is the spacetime activation function and $\rho_c$ is the critical density threshold. Under smooth gradients, this reduces to the standard MONDian interpolating function.

\section{Observational Mapping \& Degeneracy}
As established in our adversarial audit, MOSGM-I exhibits dual-mode behavior:
\begin{itemize}
    \item \textbf{Regime A (Smooth Gradients):} A substantial fraction of the parameter space exhibits degeneracy with MOND within current observational tolerances.
    \item \textbf{Regime B (High Gradients):} Non-degenerate "Goldilocks Zones" where $|\nabla \rho| \gtrsim 2~\text{kpc}^{-1}$ produce residuals that diverge from MONDian predictions.\footnote{Formal quantification of this fraction is audit-dependent and is treated as a qualitative structural property.}
\end{itemize}

\section{The AI-Orchestrated Methodology}
Unique to this study is the use of a Multi-AI Council for validation. 
\begin{quote}
    \textit{"The robustness of MOSGM-I stems from its transparency. By explicitly documenting failure modes identified by Grok and degeneracy limits verified by DeepSeek, we provide a self-audited theoretical framework."}
\end{quote}

\section{Future Falsifiability: SKA-Era}
The primary test for MOSGM-I lies in sub-kpc kinematic mapping. High-resolution HI observations from the SKA will be capable of resolving disk-edge gradients where MOSGM-I residuals should emerge. Failure to detect these gradient-correlated residuals will serve as a falsification of the current model.

\section{Conclusion}
MOSGM-I shifts the gravitational discourse from curve-fitting to mechanism-seeking. By identifying MOND as a smooth-gradient limit, we provide a testable path forward for emergent gravity theories.

\section*{Acknowledgments}
The author acknowledges the computational and adversarial audits provided by the AI Council: DeepSeek (Mathematics), Grok (Stress-testing), and Gemini (Synthesis). All final scientific judgments remain with the human lead.

\bibliographystyle{plain}
\bibliography{references}

\end{document}
